
The last decade has seen a priority shift for processor manufactures. If clock frequency
was once the main metric for performance, today computing power is measured in number of
cores in a single chip.
For software developers and computer scientists, once focused in developing sequential programs,
newer hardware usually meant faster programs without any change to the source code. Today,
the free lunch is over. Multicore processors are now forcing the development of
new software methodologies that take advantage of increasing processing power through parallelism.
However, parallel programming is difficult, usually because programs are written
in imperative and stateful programming languages that make use of low level synchronization
primitives such as locks, mutexes and barriers. This tends to make the task of managing multithreaded
execution quite intricate and error-prone, resulting in race hazards and deadlocks.
In the future, \emph{many-core} processors will make this task look even more daunting.

Advances in network speed and bandwidth are making distributed computing
more appealing. For instance, \emph{cloud computing} is a new emerging paradigm that wants
to make every computer connected to the Internet as a part of a pool of computing power,
where data can be retrieved and computation performed. From the perspective of high performance
computing, the \emph{computer cluster} is a well established paradigm that uses fast local area
networks to improve performance and solve problems that would take a long time with a single computer.

Developments in parallel and distributed programming have given birth to several programming models.
At one end of the spectrum are lower-level programming abstractions such as
\emph{message passing} (e.g., MPI~\cite{gabriel04-open-mpi}) and \emph{shared memory}
(e.g., Pthreads~\cite{Butenhof:1997:PPT:263953} or OpenMP~\cite{Chapman-2007-UOP-1370966}).
While such abstractions are very expressive and enable the programmer to write high performant code,
program APIs are very hard to use and debug, which makes it difficult to prove that a program is correct.
On the opposite end, we have many declarative programming models
that can be run in parallel~\cite{Blelloch:1996:PPA:227234.227246}.
While those declarative paradigms tend to make programs easier to reason about, they tend to offer
little or no control to the programmer for managing parallel execution
which may result in suboptimal performance.

In the context of the Claytronics project~\cite{goldstein-computer05}, Ashley-Rollman~\cite{ashley-rollman-iclp09, ashley-rollman-derosa-iros07wksp}
has created Meld, a logic programming language suited to program massively distributed systems made of modular robots with a dynamic topology.
Meld programs can derive actions that are used
to make the robots act on the outside world. The distribution of computation is done
by first partitioning the program state across the robots and then making the computation local to the node. Because Meld programs
are sets of logical clauses, they are more amenable to proof.

However, Meld and other declarative programming models give very little control to the programmer since they are stateless languages.
This is a clear disadvantage against lower-level abstractions, since its difficult to change how programs are scheduled by
the runtime system and how the system manages parallelism.

In this proposal, we present Linear Meld (LM), a new programming language for parallel programming that extends the
original Meld with linear logic in order to make programs more stateful. Linear logic gives the language a structured
way to manage state, allowing the programmer to derive and delete logical facts.
While the new language retains the declarative aspects of Meld, it adds explicit programmer control
due to the introduction of linear logic and the opportunities for optimization that arise with it.
We reuse the concept of action facts present in the original Meld in order to produce parallel scheduling changes
in the runtime system. We intend to introduce opportunities for program optimization, improved data partitioning and
parallel scheduling by using the same language constructs that are already used for standard computation.

We are interested in efficiently executing graph-based algorithms on multicores. Meld is a good
starting point because it sees a distributed program as a network of processing units, therefore
there is a naturally mapping to these kinds of algorithms.

\section{Related Work}

\subsection{Declarative Programming}

Many programming models have been developed in order to make parallel programs both easier to write and reason about. The most famous examples of such paradigms are \emph{logic programming} and \emph{functional programming}.
In logic languages such as Prolog, researchers took advantage of the non-determinism of proof-search to evaluate subgoals
in parallel with models such as \emph{or-parallelism} and \emph{and-parallelism}~\cite{Gupta:2001:PEP:504083.504085}.
In functional languages, the stateless nature of computation allows multiple expressions to evaluate safely in parallel.
This has been initially explored in several languages, such as NESL~\cite{Blelloch:1996:PPA:227234.227246} or Id~\cite{Nikhil93anoverview}, and later implemented in more modern languages such as Haskell~\cite{Chakravarty07dataparallel}.

Recently, there has been an increasing interest in declarative and data-centric languages.
MapReduce~\cite{Dean:2008:MSD:1327452.1327492}, for instance, is a popular data-centric programming
model that is optimized for large clusters. The scheduling and data sharing model is very simple:
in the \emph{map phase}, data is transformed at each node and the result reduced to a final
result in the \emph{reduce phase}.

Another declarative approach that is regaining popularity is Datalog~\cite{Ullman:1990:PDK:533142}, a
bottom-up logic programming language that was the inspiration for the original Meld.
Traditionally used in deductive databases, it is now being increasingly used in different fields
such as distributed networking~\cite{Loo-condie-garofalakis-p2}, sensor
nets~\cite{Chu:2007:DID:1322263.1322281} and cloud computing~\cite{alvaro:boom}.

\subsection{Graph-Based Programming Models}

Like Meld, many programming systems also model the program as a graph where computation will be performed.
The Dryad system~\cite{Isard:2007:DDD:1272996.1273005} combines computational vertices
with communication channels (edges) to form a data-flow graph. The program is scheduled to
run on multiple computers or cores and data is partitioned during runtime. Routines that run on computational vertices
are sequential, with no synchronization.

The Pregel system~\cite{Malewicz:2010:PSL:1807167.1807184} is also graph based, although programs have a more strict
structure. They must be represented as a sequence of iterations where each iteration is composed of computation and message passing.
Pregel is specially suited to solve very big graphs
and to scale to large architectures.

GraphLab~\cite{GraphLab2010} is a C++ framework for developing parallel machine learning
algorithms. While Pregel uses message passing, GraphLab allows nodes to have read/write
access to different scopes through different concurrent access models in order to balance
performance and data consistency. While some programs only need to access the local node's
data, others may need to update edge information. Each consistency model will provide different
guarantees that are better adapted to some algorithms. GraphLab also provides different
schedulers that dictate the order in which node's are computed.

\subsection{Linear Logic}

Linear logic is a substructural logic proposed by Jean-Yves Girard~\cite{Girard95logic:its} that extends intuitionistic logic with the concept of \emph{truth as resources}. Instead of seeing the truth as immutable, truth is now something that can be consumed during the proof process.

Since computer science is focused on processes and algorithms, linear logic has been used
in many areas of computing such as programming languages, game semantics, concurrent programming, knowledge representation, etc.
Due to the resource interpretation of the logic, linear logic presents a good basis for programming
languages that allow state manipulation.

In the context of the Curry-Howard correspondence~\cite{howard:tfatnoc}, linear logic has been applied in programming languages
as a mechanism to implement \emph{linear types}. Linear types or also sometimes known as \emph{uniqueness types} are types
that force objects to be used exactly once. Surprisingly, such types add mutable state to functional languages because they enforce
a linear view of state, allowing the language to naturally support concurrency, input/output and data structure's updates.
Arguably, the most popular language that features uniqueness types is the Clean programming language~\cite{JFP:1349748}.
Monads~\cite{Wadler:1997:DI:262009.262011}, made popular with the Haskell programming language, are another interesting way to add state
to functional languages. While monads tend to be more powerful than linear types, they also ensure equational reasoning in the presence
of mutable data structures and I/O effects.

As we will see in the next chapters of this proposal, linear logic programming is a different approach than either monads or linear types.
While the latter are mechanisms that enhance functional programming with state, the former uses state as a foundation, since
the program's database is both the state and the program, since it drives the computation forward through rule application. However, our
new language can also interact with the outside world through sensing and action facts, which are special facts that return information
about the world and act on the outside world, respectively.

\subsection{Coordination}

FIXME

\subsection{Provability}

Many techniques and formal systems have been devised to help reason about parallel programs.
One such example is the Owicki-Gries~\cite{Owicki:1976:VPP:360051.360224} deductive system
for proving properties about imperative parallel programs (deadlock detection, termination, etc).
It extends Hoare logic with a stronger set axioms such as parallel execution, critical section
and auxiliary variables. The formal system can be successfully used in small imperative
programs, although using it on languages such as C is difficult since they do not
restrict the use of shared variables.

Some formal systems do not build on top of a known programming paradigm, but instead
create an entirely new formal system for describing concurrent systems. Process calculus
such as $\pi$-calculus~\cite{Milner:1999:CMS:329902} is a good example of this.
The $\pi$-calculus describes the interactions between processes
through the use of channels for communication. Interestingly, channels can also be transmitted as
messages, allowing for changes in the network of processes.
Given two processes, $\pi$-calculus is able to prove that they behave the same through
the use of bisimulation equivalence.

Another interesting model is Mobile UNITY~\cite{Roman97anintroduction}. The basic UNITY~\cite{UNITY} model assumes that statements could be executed non-deterministically
in order to create parallelism. This principle is applied to prove properties about
the system.
Mobile UNITY transforms UNITY by adding locations to processes and removing the
nondeterministic aspect from local processes. Processes could then communicate or move
between locations.

Since the original Meld is based on logic programming, it has been used to produce proofs of correctness.
Meld program code is amenable to mechanized analysis via theorem checkers such as Twelf~\cite{twelf},
a logic system designed for analyzing program logics and logic program implementations.
For instance, a meta-module based shape planner program was proven to be correct~\cite{dewey-iros08,ashley-rollman-iclp09}
under the assumption that actions derived by the program are always successfully applied in the outside world.
While the fault tolerance aspect is lax, the planner will always reach the target shape in finite time.
The sketch of the proof is presented in Dewey et al~\cite{dewey-iros08}.

\section{Thesis Statement}


We propose Linear Meld (\lang), a new linear logic programming language that is a suitable
to efficiently execute and scale parallel programs on multicore architectures.
We argue that \lang is a suitable declarative programming model that offers execution
control to the programmer through the use of coordination directives to make the program
faster and more scalable. These coordination
directives change how the runtime system schedules computation and can be written with the same
facilities used to write standard program code. Finally, we want to show that the logical foundations
of \lang can be leveraged to make program execution competitive when compared to implementations
in other declarative languages.

\lang will accomplish these goals using four main ideas:


\begin{itemize}
   \item Implicit Parallelism (mostly done)
   
   We divide the logical facts across all the nodes of the graph. Since the
   logical rules only make use of data local to a node, computation can be performed at the
   node level, disregarding the remaining nodes of the graph. We envision the program as
   a graph data structure where each processing unit performs work on a different subset of the graph,
   thus creating parallelism. Communication between processing units only happens when nodes send data
   to nodes in other threads.
   
   \item Linear Logic (mostly done)

   We integrated linear logic~\cite{Girard95logic:its} into our language, so that program state
   can be encoded naturally. The original Meld was fully based on classical logic where everything that
   is derived to be true forever. Linear logic turns some facts into resources that will be consumed when a rule is applied. We can leverage this sound logical foundation to prove many properties about our programs, including correctness and termination. To the best of our knowledge, \lang is the first
   linear logic based language implementation that attempts to solve real world problems.

   \item Coordination (partly done)
   
   We are using the concept of \emph{action facts} to coordinate the execution of programs.
   We can increase the priority of certain nodes during runtime according to the state of the
   computation and to the state of the runtime in order to make better scheduling decisions
   so that programs can run faster.
   For example, consider the shortest path program. We can pick nodes with shorter
   distances to the source node before other nodes, so that convergence is reached faster.
   We also use action facts to model output and to visualize the program's behavior in the
   interfaces that we have built. We intend to add more coordination directives and action facts
   and also write more programs that take advantage of coordination.

   \item Implementation (partly done)

   We have implemented a new compiler and a virtual machine prototype from scratch that executes on multicore machines\footnote{Source code is available at \url{http://github.com/flavioc/meld} (virtual machine) and \url{http://github.com/flavioc/cl-meld} (compiler).}.
   To test the robustness and efficiency of our implementation, several different
   and interesting programs were coded in \lang such as belief propagation~\cite{Gonzalez+al:aistats09paraml},
   belief propagation with residual splash~\cite{Gonzalez+al:aistats09paraml}, PageRank~\cite{Page:2001:MNR},
   graph coloring~\cite{PSP:2032868}, N queens~\cite{8queens}, shortest path~\cite{Dijkstra}, diameter estimation~\cite{5234320}, MapReduce~\cite{Dean:2008:MSD:1327452.1327492}, game of life, quick-sort, neural network training, among others. Our results show that \lang makes programs scalable, although it falls short when compared against other systems in absolute run time. We propose more optimization work
   in order to make \lang more competitive against other systems. We want to leverage the sound logical
   foundations of our language to optimize the compiler and runtime system.
   
\end{itemize}

