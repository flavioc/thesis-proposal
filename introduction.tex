
The last decade has seen a priority shift for processor manufactures. If clock frequency
was once the main metric for performance, today computing power is measured in number of
cores in a single chip.
For software developers and computer scientists, once focused in developing sequential programs,
newer hardware usually meant faster programs without any change to the source code. Today,
the free lunch is over. Multicore processors are now forcing the development of
new software methodologies that take advantage of increasing processing power through parallelism.
However, parallel programming is difficult, usually because programs are written
in imperative and stateful programming languages that make use of low level synchronization
primitives such as locks, mutexes and barriers. This tends to make the task of managing multithreaded
execution quite intricate and error-prone, resulting in race hazards and deadlocks.
In the future, \emph{many-core} processors will make this task look even more daunting.

Advances in network speed and bandwidth are making distributed computing
more appealing. For instance, \emph{cloud computing} is a new emerging paradigm that wants
to make every computer connected to the Internet as a part of a pool of computing power,
where data can be retrieved and computation performed. From the perspective of high performance
computing, the \emph{computer cluster} is a well established paradigm that uses fast local area
networks to improve performance and solve problems that would take a long time with a single computer.

Developments in parallel and distributed programming have given birth to several programming models.
At one end of the spectrum are lower-level programming abstractions such as
\emph{message passing} (e.g., MPI~\cite{gabriel04-open-mpi}) and \emph{shared memory}
(e.g., Pthreads~\cite{Butenhof:1997:PPT:263953} or OpenMP~\cite{Chapman-2007-UOP-1370966}).
While such abstractions are very expressive and enable the programmer to write high performant code,
program APIs are very hard to use and debug, which makes it difficult to prove that a program is correct.
On the opposite end, we have many declarative programming models
that can be run in parallel~\cite{Blelloch:1996:PPA:227234.227246}.
While those declarative paradigms tend to make programs easier to reason about, they tend to offer
little or no control to the programmer for managing parallel execution
which may result in suboptimal performance.

In the context of the Claytronics project~\cite{goldstein-computer05}, Ashley-Rollman et al.~\cite{ashley-rollman-iclp09, ashley-rollman-derosa-iros07wksp}
has created Meld, a logic programming language suited to program massively distributed systems made of modular robots with a dynamic topology.
Meld programs can derive actions that are used
to make the robots act on the outside world. The distribution of computation is done
by first partitioning the program state across the robots and then making the computation local to the node. Because Meld programs
are sets of logical clauses, they are more amenable to proof.

However, Meld and other declarative programming models give very little control to the programmer since they are stateless languages.
This is a clear disadvantage against lower-level abstractions, since its difficult to change how programs are scheduled by
the runtime system and how the system manages parallelism.

In this proposal, we present Linear Meld (LM), a new language for parallel programming that extends the
original Meld with linear logic. Linear logic gives the language a structured
way to manage state, allowing the programmer to derive and delete logical facts.
While the new language retains the declarative aspects of Meld due to the introduction of linear logic, but also adds explicit programmer control and opportunities for optimization that arise with stateful programs.
We reuse the concept of action facts, present in the original Meld, in order to pass scheduling decisions to
the runtime system. We intend to introduce opportunities for program optimization, improved data partitioning and
parallel scheduling by using the same logical code used for standard computation.

In particular, we are interested in efficiently executing graph-based algorithms on multicores. The original Meld is a good
starting point because it sees a distributed program as a network of processing units, therefore
there is a naturally mapping to these kinds of algorithms.

\section{Thesis Statement}


We propose LM, a new linear logic programming language designed
to efficiently execute and scale parallel graph based programs on multicore architectures.
We argue that LM is a suitable declarative programming model and that the logical foundations
of LM can be leveraged to make programs more expressive and faster to execute when compared to implementations
in other declarative languages.
We will prove our thesis through five major goals:

\begin{itemize}
   
   \item Linear Logic (mostly done)

   We integrated linear logic~\cite{Girard95logic:its} into our language, so that program state
   can be encoded naturally. The original Meld was fully based on classical logic where everything that
   is derived is true forever. Linear logic turns some facts into resources that will be consumed when a rule is applied.
   We can leverage this sound logical foundation to prove many properties about our programs, including correctness and termination.
   To the best of our knowledge, \lang is the first
   linear logic based language implementation that attempts to solve real world problems.

   \item Coordination (partly done)
   
   LM offers execution control to the programmer through the use of coordination directives to make the program
   faster and more scalable. These coordination
   directives change how the runtime system schedules computation and can be written with the same
   facilities used to write standard program code.
   We are using the concept of \emph{action facts} to coordinate the execution of programs.
   We can increase the priority of certain nodes during runtime according to the state of the
   computation and to the state of the runtime in order to make better scheduling decisions
   so that programs can run faster.
   For example, consider the shortest path program. We can pick nodes with shorter
   distances to the source node before other nodes, so that convergence is reached faster.
   We also use action facts to model output and to visualize the program's behavior in the
   interfaces that we have built. We intend to add more coordination directives and action facts
   and also write more programs that can take advantage of coordination.

   \item Fast Sequential Execution (partly done)
   
   Since LM uses logical rules to perform computation, many program optimizations are possible. The use of linear logic
   opens new opportunities for code improvement since linear logic has some similarities with imperative programming.
   We have already explored some potential optimizations in our implementation, including detecting cases where a fact
   is re-derived with modified arguments. We intend to explore further optimizations, including whole-program optimizations.
   
   \item Multicore Parallelism (partly done)
   
   We divide the logical facts across all the nodes of the graph. Since the
   logical rules only make use of data local to a node, computation can be performed at the
   node level, without reference from other nodes of the graph. We envision the application as
   a communicating graph data structure where each processing unit performs work on a different subset of the graph,
   thus creating parallelism. This is an advantage of LM since we can run programs on many different types
   of distributed systems as long as the underlying runtime system uses the appropriate communication facilities.

   \item Experimental Results (partly done)

   We have implemented a new compiler and a virtual machine prototype from scratch that executes on multicore machines\footnote{Source code is available at \url{http://github.com/flavioc/meld} (virtual machine) and \url{http://github.com/flavioc/cl-meld} (compiler).}.
   We have implemented programs such as belief propagation~\cite{Gonzalez+al:aistats09paraml},
   belief propagation with residual splash~\cite{Gonzalez+al:aistats09paraml}, PageRank~\cite{Page:2001:MNR},
   graph coloring~\cite{PSP:2032868}, N queens~\cite{8queens}, shortest path~\cite{Dijkstra}, diameter estimation~\cite{5234320}, MapReduce~\cite{Dean:2008:MSD:1327452.1327492}, game of life, quick-sort, neural network training, among others.
   Our preliminary results show that our particular implementation makes programs scalable with up to 16 threads. We intend to
   further improve the scalability of our virtual machine.
   
\end{itemize}


