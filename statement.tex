
We propose Linear Meld (LM), a new linear logic programming language that is a suitable
to efficiently execute and scale parallel graph based programs on multicore architectures.
We argue that \lang is a suitable declarative programming model that offers execution
control to the programmer through the use of coordination directives to make the program
faster and more scalable. These coordination
directives change how the runtime system schedules computation and can be written with the same
facilities used to write standard program code. Finally, we want to show that the logical foundations
of \lang can be leveraged to make program execution competitive when compared to implementations
in other declarative languages.

\lang will accomplish these goals using four main ideas:


\begin{itemize}
   \item Implicit Parallelism (mostly done)
   
   We divide the logical facts across all the nodes of the graph. Since the
   logical rules only make use of data local to a node, computation can be performed at the
   node level, disregarding the other nodes of the graph. We envision the program as
   a graph data structure where each processing unit performs work on a different subset of the graph,
   thus creating parallelism. Communication between processing units only happens when nodes send data
   to nodes in other units.
   
   \item Linear Logic (mostly done)

   We integrated linear logic~\cite{Girard95logic:its} into our language, so that program state
   can be encoded naturally. The original Meld was fully based on classical logic where everything that
   is derived is true forever. Linear logic turns some facts into resources that will be consumed when a rule is applied. We can leverage this sound logical foundation to prove many properties about our programs, including correctness and termination. To the best of our knowledge, \lang is the first
   linear logic based language implementation that attempts to solve real world problems.

   \item Coordination (partly done)
   
   We are using the concept of \emph{action facts} to coordinate the execution of programs.
   We can increase the priority of certain nodes during runtime according to the state of the
   computation and to the state of the runtime in order to make better scheduling decisions
   so that programs can run faster.
   For example, consider the shortest path program. We can pick nodes with shorter
   distances to the source node before other nodes, so that convergence is reached faster.
   We also use action facts to model output and to visualize the program's behavior in the
   interfaces that we have built. We intend to add more coordination directives and action facts
   and also write more programs that take advantage of coordination.

   \item Implementation (partly done)

   We have implemented a new compiler and a virtual machine prototype from scratch that executes on multicore machines\footnote{Source code is available at \url{http://github.com/flavioc/meld} (virtual machine) and \url{http://github.com/flavioc/cl-meld} (compiler).}.
   To test the robustness and efficiency of our implementation, several different
   and interesting programs were coded in \lang such as belief propagation~\cite{Gonzalez+al:aistats09paraml},
   belief propagation with residual splash~\cite{Gonzalez+al:aistats09paraml}, PageRank~\cite{Page:2001:MNR},
   graph coloring~\cite{PSP:2032868}, N queens~\cite{8queens}, shortest path~\cite{Dijkstra}, diameter estimation~\cite{5234320}, MapReduce~\cite{Dean:2008:MSD:1327452.1327492}, game of life, quick-sort, neural network training, among others. Our results show that \lang makes programs scalable, although it falls short when compared against other systems in absolute run time. We propose more optimization work
   in order to make \lang more competitive against other systems. We also want to leverage the sound logical
   foundations of our language to optimize the compiler and runtime system.
   
\end{itemize}
